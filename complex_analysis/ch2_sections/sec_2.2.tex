\section{Convergent Power Series}

\subsection{Sequences and Series of Complex Numbers}

\begin{defn}[\textbf{Sequence}]
  \begin{align*}
    { \{ z_n \} } \;\; or \;\; z_n
  \end{align*}
\end{defn}

\begin{defn}[\textbf{Series}]
  \begin{align*}
    \sum_{n = 1}^{\infty} z_n
  \end{align*}
\end{defn}

\begin{defn}[\textbf{Partial Sum}]
  \begin{align*}
    s_n = \sum_{k = 1}^{n} z_k = z_1 + z_2 + \cdots z_n
  \end{align*}
\end{defn}

\subsection{Convergence of Complex Numbers}
\begin{defn}[\textbf{Convergent Series}]
  The series $s_n$ converges if there is some $w \in \mathbb{C}$ such that the following limit exists:
  \begin{align*}
    \lim_{n \to \infty}s_n = w
  \end{align*}
\end{defn}

In which case we say that $w$ is equal to the \emph{sum of the series.}

\begin{defn}[\textbf{Sum of the Series}]
  Formally:
  \begin{align*}
    w = \sum_{n = 1}^{\infty} z_n
  \end{align*}
\end{defn}

Now consider \emph{} and deal with questions of \emph{uniformity.}

\subsection{Sequences of Functions}
Suppose:
\begin{itemize}
  \item $S \subseteq \mathbb{C}$
  \item $f$ is a \emph{bounded function} on $S$
  \item $n \in \mathbb{N}$
\end{itemize}

\begin{defn}[\textbf{Sequence}]
  \begin{align*}
    \{ f_n \} \; or \; f_n 
  \end{align*}
\end{defn}

\begin{defn}[\textbf{Sup norm}]
  Let sup be the least upper bound, then:
  \begin{align*}
    \| f \|_S = \| f \| = \sup_{z \in S} | f(z) | \\
  \end{align*}
\end{defn}

\subsection{Uniform Convergence of Complex Functions}

\begin{defn}[\textbf{Uniform Convergence}]
  The sequence $ f_n $ converges uniformly on $S$ if there exists a 
  function $f$ on $S$ satisfying the following properties.

  Given $\epsilon$, there exists $N$ such that if $n \geq N$ then:

  \begin{align*}
    \| f_n - f \| < \epsilon \\
  \end{align*}
\end{defn}

\begin{defn}[\textbf{Cauchy Sequence}]
  We call $f_n$ Cauchy if given $\epsilon$ there exists $N$ such that if $m, n \geq N$ then:
  \begin{align*}
    \| f_n - f_m \| < \epsilon
  \end{align*}
\end{defn}

\begin{thm}
  If a sequence $f_n$ of functions on $S$ is Cauchy, then $f_n$ converges uniformly.
\end{thm}

\begin{thm}
  If the functions $f_n$ in the theorem are bounded, then the limiting function $f$ is bounded.
\end{thm}

\subsection{Series of Functions}

\begin{defn}[\textbf{Partial Sum}]
  \begin{align*}
    s_n = \sum_{k = 1}^{n} f_k = f_1 + f_2 + \cdots f_n
  \end{align*}
\end{defn}

\begin{defn}[\textbf{Uniform Convergence of Series}]
  The series $\sum_{k = 1}^{n} f_k$ converges uniformly if the sequence of partial 
  sums $s_n$ converges uniformly.
\end{defn}

\begin{defn}[\textbf{Absolute Convergence of Series}]
  A series $\sum f_n$ converges absolutely if $\forall z \in S$ the following series converges:
  \begin{align*}
    \sum | f_n (z) |
  \end{align*}
\end{defn}

\begin{thm}[\textbf{Comparison Test}]
  Let $c_n$ be a sequence of real numbers $\geq 0$ and assume that
  \begin{align*}
    \sum c_n
  \end{align*}
  converges. Let $f_n$ be a sequence of functions on S such that 
  $||f_n|| \leq c_n$ for all n. Then $\sum f_n$ converges uniformly and absolutely.
\end{thm}

\subsubsection*{Power Series}

Now consider the functions $f_n$ in the form
\begin{align*}
  a_n \in \mathbb{C} \;\; n \in \mathbb{N} \\
  \\
  f_n (z) = a_n z^n \\ 
\end{align*}

\begin{thm}[\textbf{Absolute Convergence}]
  Let $a_n$ be a sequence of complex numbers, and let $r$ be a number $> 0$ such that the series 
  \begin{align*}
    \sum | a_n | r^n
  \end{align*}
  converges. Then for $|z| \leq r$ the series $\sum a_n z^n$ converges absolutely and uniformly. 
\end{thm}

\begin{thm}[\textbf{Existence of the Radius of Convergence}]
  Let $\sum a_n z^n$ be a power series. If it does not converge absolutely for all $z$, then 
  there exists a number $r$ such that the series converges absolutely for $|z| < r$ and does not 
  converge for $|z| > r.$
\end{thm}

\begin{defn}[\textbf{Radius of Convergence} $r$]
  The number $r \not = |z|$ such that  $\sum a_n z^n$ converges for $|z| < r.$
  \begin{itemize}
    \item If the power series converges for all $z$ then the radius of convergence is infinity.
    \item If the radius of convergence is $r = 0$ then the series converges absolutely only for $z = 0.$
    \item The radius of convergence can be determined by the coefficients.
  \end{itemize}
\end{defn}

\begin{defn}[\textbf{Convergent Power Series}]
  Any power series that has a non-zero radius of convergence.
\end{defn}

\begin{defn}[\textbf{Power Series Converging on a Disk} $D$]
  If $D$ is a disk centered at the origin contained in the disk $\mathbb{D}(0, r)$, where $r$ is the 
  radius of convergence for some power series g, then g converges on $D.$ 
\end{defn}

\begin{defn}[\textbf{Point of Accumulation}]
  Given a sequence $t_n$ of real numbers, the number $t$ is the point of accumulation of $t_n$ if 
  given $\epsilon,$ there exist infinitely many indices $n$ such that
  \begin{align*}
    |t_n - t| < \epsilon
  \end{align*}
\end{defn}
So, infinitely many points of the sequence lie in a given interval centered at $t.$ The Weierstrass-Balzano
theorem proves that, \emph{in the reals}, every bounded sequence has a point of accumulation. There can 
be many points of accumulation as well.

\textbf{Assume}
\begin{itemize}
  \item $t_n$ is \emph{bounded}
  \item Let $S$ be the set of \emph{points of accumulation}
\end{itemize}

\begin{defn}[\textbf{Limit Superior}: lim sup $t_n$]
  The least upper bound of $S.$
\end{defn}

Let $\lambda \in S$ be this least upper bound, then one writes: $\lambda = $ lim sup $t_n =$
least upper bound of $S.$

Notice that $\lambda$ has the following properties:
\begin{thm}[\textbf{Epsilon Around Lamda}]
  Given $\epsilon,$ there exist only finitely many $n$ such that
  \begin{align*}
    t_n \geq \lambda + \epsilon
  \end{align*}
  There exist infinitely many $n$ such that
  \begin{align*}
    t_n \geq \lambda - \epsilon
  \end{align*}
\end{thm}

\newpage