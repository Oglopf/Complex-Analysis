\section{Relations Between Formal and Convergent Series}

\subsubsection*{Sums and Products}
Let $f = f(T)$ and $g = g(t)$ be formal power series.

If $f$ converges absolutely for some complex number $z$, then we have the value $f(z)$ and similarly for $g(z).$

And so the big theorem here is that if $f$ and $g$ converge on the disc, then so do their sum and products.

\begin{defn}[\bfseries Logarithm Series of Real x]
    For real values of $x$ we have:
    \begin{align*}
        f(x)         & = \log (1 + x)                                   \\
        \log (1 + x) & = \sum_{n=1}^{\infty} (-1)^{n - 1} \frac{x^n}{n}
    \end{align*}
\end{defn}

And this then results in the logarithm for a complex number by:
\begin{defn}[\bfseries Logarithm Series of Complex z]
    Let $|z - 1| < 1$ and define $\log z$ as:
    \begin{align*}
        \log z                 & = f(z - 1)                                               \\
        f(z - 1)               & = \log(1 + (z - 1))                                      \\
        \\
        \therefore \;\; \log z & = \sum_{n=1}^{\infty} (-1)^{n - 1} \frac{({z - 1})^n}{n}
    \end{align*}
\end{defn}
