\section{The Cauchy-Reimann Equations}
In this section:
\begin{itemize}
	\item Let $f$ be a function on an open set $U.$
	\item Write $f$ in terms of its \textit{real} and \textit{imaginary} parts.
	\begin{align*}
		f(x + iy) = u(x, y) + iv(x, y) \\
	\end{align*}
	\item We derive the equivalent conditions on $u$ and $v$ for $f$ to be holomorphic.
\end{itemize}

At a fixed $z$: 
\begin{itemize}
	\item let $f'(z) = a + bi.$
	\item let $w = h + ik \,\,\,\, h, k \in \mathbb{R}$
	\item Suppose:
	\begin{align*}
		\lim_{w \to 0} \sigma(w) &= 0 \\
		f(z + w) - f(z) &= f'(z)w + \sigma(w)w \\
	\end{align*}
	\item Let:
	\begin{align*}
		\vec{F}: U \to \mathbb{R}^2 \\
	\end{align*}
	\item such that:
	\begin{align*}
		\vec{F}(x, y) &= (u(x, y), v(x, y))
	\end{align*}
\end{itemize}

\begin{itemize}
		\item We call $\vec{F}$ the (real) \textbf{Field Associated with} $f.$ 
		\item If we assume that $f$ is holomorphic then $\vec{F}$ is differentiable, and its derivative is represented by the \textbf{Jacobian Matrix}:
		\begin{align*}
			J_{\vec{F}}(x, y) = \left( \begin{matrix}
				a & -b \\
				b &\,\,\,\,  a 
			\end{matrix}
			\right)
			= \left( \begin{matrix}
				\frac{\partial u}{\partial x} & \frac{\partial u}{\partial y} \\
				\frac{\partial v}{\partial x} & \frac{\partial v}{\partial y}
			\end{matrix} \right)
		\end{align*}
		\item This shows:
		\begin{align*}
			f'(z) = \frac{\partial u}{\partial x} - i\frac{\partial u}{\partial y}
		\end{align*}
\end{itemize}
This all culminates to the incredibly important result of:
\begin{defn}[\textbf{Cauchy-Riemann Equations}]
	\begin{align*}
		\frac{\partial u}{\partial x} = \frac{\partial v}{\partial y} \,\,\,\, \text{and} \,\,\,\, \frac{\partial u}{\partial y} = -\frac{\partial v}{\partial x}
	\end{align*}	
\end{defn}


Last bit over Jacobian Determinant $\triangle_{\vec{F}}$ needed.

\newpage