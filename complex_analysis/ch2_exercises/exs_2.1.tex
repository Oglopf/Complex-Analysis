\subsection{Proofs}
\begin{enumerate}

    \item Give the terms of order $\leq 3$ in the power series:
    The trick for all of these is to just use their power series expansions or use ones we've found and stuff more $f$s into them 
    to get out more that one needs.
    \begin{enumerate}
      \item $e^z \sin(z)$
      \item $\sin z \cos z$
      \item $\frac{e^z - 1}{z}$
      \item $\frac{e^z - \cos z}{z}$
      \item $\frac{1}{\cos z}$
      \item $\frac{\cos z}{\sin z}$
      \item $\frac{\sin z}{\cos z}$
      \item $\frac{e^z}{\sin z}$
    \end{enumerate}
    
    \item Let $f(z) = \sum a_n z^n$.
  
    Define $f(-z) = \sum a_n (-z)^n = \sum a_n(-1)^n z^n$.
  
    Define $f$ as \textbf{even} if $a_n = 0$ for $n$ odd.
  
    Define $f$ as \textbf{odd} if $a_n = 0$ for $n$ even.
  
    Verify that $f$ is even if and only if $f(-z) = f(z)$ and $f$ is odd if and only if $f(-z) = -f(z).$
  
    \textbf{Proof}:
    Use power series expansions. Proving the \textbf{even} case first, but the same type of argument will work for the odd.
  
    If $f$ is \textbf{even} then $n$ is odd, so:
    \begin{align*}
      f(-z) &= \sum a_n z^n (-1)^n \\
      &= a_1 z (-1) + a_2 z^2 (-1)^2 + a_3 z^3 (-1)^3 \\
      &+ a_4 z^4 (-1)^4 + a_5 z^5 (-1)^5 + a_6 z^6 (-1)^6 + \cdots + a_n z^n (-1)^n \\
    \end{align*}
  
    Recall that since this is \textbf{even} then all the $a_n = 0$ for odd $n$ which leaves us 
    with (note $k \in \mathbb{N}$ here):
    \begin{align*}
      f(-z) &= a_2 z^2 (-1)^2 + a_4 z^4 (-1)^4 + a_6 z^6 (-1)^6 + \cdots + a_n z^{2k} (-1)^{2k} \\
    \end{align*} 
    Noticing that all of the factors of $(-1)$ have the form $(-1)^{2k}$ one sees:
    \begin{align*}
      f(-z) = f(z)
    \end{align*}
    Conversely, if you assume $f(-z) = f(z)$ then $f$ is \textbf{even} simply by definition, 
    which completes the proof for the \textbf{even} case. 
    
    A similar argument can be applied for the \textbf{odd} case.
    \qed
  
    \item Define the \textbf{Bernoulli numbers} $B_n$ by the power series:
  
    \begin{align*}
      \frac{z}{e^z - 1} = \sum_{n = 0}^{\infty}\frac{B_n}{n\,!}z^n \\
    \end{align*}
  
    Prove the recursion formula:
  
    \begin{align*}
      \frac{B_0}{n \, ! \; 0!} + \frac{B_1}{(n - 1)! \; 1!} + \cdots + \frac{B_{n - 1}}{1! \; (n - 1)\,!} + = 
      \begin{cases}
        1 \;\;\; \text{   if } n = 1. \\
        0 \;\;\; \text{   if } n > 1. \\
      \end{cases}
    \end{align*}
    Then $B_0 = 1.$
    
    Compute $B_1, B_2, B_3, B_4.$
    
    Show that $B_n = 0$ if $n$ is odd and $n \not = 1.$

    \textbf{Proof}:
    First, transform the given function into a geometric series, then expand the series to find the coefficients
    $a_n$ for the terms, which will correspond to $a_n n! = B_n$ in the recursion formula. 

    \begin{align*}
      \frac{z}{e^z - 1} &= \frac{z}{-1 + e^z} \\
      &= \frac{z}{-1 + \left(1 + z + \frac{z^2}{2!} + \cdots + \frac{z^n}{n!} \right)} \\
      &= \frac{z}{\left(z + \frac{z^2}{2!} + \cdots + \frac{z^n}{n!} \right)} \\
      &= \frac{1}{\left(1 + \frac{z}{2!} + \cdots + \frac{z^{n - 1}}{n!} \right)} \\
      &= \frac{1}{1 - \left( - \frac{z}{2!} + \cdots + \frac{z^{n - 1}}{n!} \right)} \\
      &= 1 + \left( - \frac{z}{2!} + \cdots + \frac{z^{n - 1}}{n!} \right) + \left( - \frac{z}{2!} + 
      \cdots + \frac{z^{n - 1}}{n!} \right)^2 + \cdots + \text{higher terms} \\
    \end{align*}
    
    Now use the terms above to solve for the needed coefficients. 
    
    Then:

    \begin{align*}
      a_0 = 1 \\
    \end{align*} 
    
    There is only one term with base $z$ which is $\frac{-1}{2!}$. So:

    \begin{align*}
      a_1 = -1/2 \\
    \end{align*}

    For $a_2$ check the $z^2$ base's coefficients.

    Distribute out the squared term to get the $z^2$ based coefficients:

    \begin{align*}
      \left( - \frac{z}{2!} + \cdots + \frac{z^{n - 1}}{n!} \right)^2 &= \left( - \frac{z}{2!} + \cdots + \frac{z^{n - 1}}{n!} \right)
      \left( - \frac{z}{2!} + \cdots + \frac{z^{n - 1}}{n!} \right)  \\
    \end{align*}

    Distribute out and notice how we need nothing above $z^2$ and can stop once we have all its terms:

    \begin{align*}
      \left( - \frac{z}{2!} + \cdots + \frac{z^{n - 1}}{n!} \right)^2 &= \left( - \frac{z}{2!} + \cdots + \frac{z^{n - 1}}{n!} \right)
      \left( - \frac{z}{2!} + \cdots + \frac{z^{n - 1}}{n!} \right)  \\
      &= \left( - \frac{z^2}{2!2!} + \text{(terms > $z^2$)} \right) \\
    \end{align*}

    Combining the matching based terms results in:
    
    \begin{align*}
      a_2 &= \frac{-1}{6} + \frac{1}{4} \\
      &= \frac{1}{12} \\
    \end{align*}

    Keeping this up results in the first four coefficients coming out to:

    \begin{align*}
      a_0 &= 1 \\
      a_1 &= \frac{-1}{2} \\
      a_2 &= \frac{1}{12}  \\
      a_3 & = \frac{-1}{24} + \frac{1}{12} + \frac{1}{12} - \frac{1}{8} \\
      &= 0 \\
    \end{align*}

    Now using the relationship given the first four Bernoulli Numbers $B_n$ come out to:

    \begin{align*}
      B_n &= a_n \cdot n! \\
      B_0 &= 1 \cdot 0! \\
      B_1 &= \frac{-1}{2} \cdot 1! \\
      B_2 &= \frac{1}{12} \cdot 2! \\
      &= \frac{1}{6} \\
      B_3 &= 0 \cdot 3! \\
      &= 0 \\
    \end{align*}

    Plugging these into the recursion results in a true statements. So, the base case is proved, assume true for $n$,
    and now show it's true for $n + 1.$

    \item Show that:
  
    \begin{align*}
      \frac{z}{2} \frac{e^{z/2} + e^{-z/2}}{e^{z/2} - e^{-z/2}} = \sum_{n = 0}^\infty \frac{B_n}{(2n)!}z^{2n} \;. \\
    \end{align*}
  
    Replace $z$ by $2\pi i z$ to show that:
  
    \begin{align*}
      \pi z \cot(\pi z) = \sum_{n = 0}^\infty (-1)^n \frac{(2\pi)^{2n}}{(2n)!} z^{2n} B_{2n} \;.
    \end{align*}
  
    \item Express the power series for $\tan z , \frac{z}{\sin z}, z \cot z ,$ in terms of \textbf{Bernoulli numbers}.
  
    \item (\textbf{Difference Equation}) given complex numbers $a_0, a_1, u_1, u_2 \;$ define $a_n$ for $2 \leq n$ by:
  
    \begin{align*}
      a_n = u_1 a_{n - 1} +u_2 a_{n - 2} \;.
    \end{align*}
  
    If you have a factorization: \textbf{STOPPED}
  \end{enumerate}
  \newpage