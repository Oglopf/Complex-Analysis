\subsection{Proofs}
\begin{enumerate}
	\item
	      \subitem a. Let $\alpha$ be a complex number of absolute value $< 1.$
	      What is $\lim\limits_{n \to \infty} \alpha^n ?$ Proof?

	      \textbf{Proof:}
	      Just use a Cauchy Sequence.

	      $|\alpha| < 1$ iff $ |\alpha| = \frac{1}{|z|}.$

	      So then check $\lim\limits_{n \to \infty} |\alpha|^n$:
	      \begin{align*}
		      \lim\limits_{n \to \infty} |\alpha|^n & = \lim\limits_{n \to \infty} \frac{1}{|z|^n} \\
	      \end{align*}
	      This sequence has different behavior for $z < 1$ vs $z > 1$, but we must have $|\alpha| < 1$ therefore $z > 1$ only here.

	      \begin{align*}
		      \lim\limits_{n \to \infty} |\alpha|^n & = \lim\limits_{n \to \infty} \frac{1}{|z|^n} \\
	      \end{align*}

	      Since $z > 1$ and by the Archimedean principal we can always find $m > n$
	      then clearly:
	      \begin{align*}
		      \alpha^m < \alpha^n \iff \frac{1}{z^m} < \frac{1}{z^n}
	      \end{align*}

	      Therefore this sequence is monotonic, it only decreases each new term, and the difference between $m, n$ terms is
	      small but always $> 0$ which gives us a cauchy sequence.

	      So then:
	      \begin{align*}
		      |\alpha^n - \alpha^m| < \epsilon \\
	      \end{align*} \qed

	      \subitem b. Let $\alpha$ be a complex number of absolute value $> 1.$
	      What is $\lim\limits_{n \to \infty} \alpha^n ?$ Proof?

	      \textbf{Proof:}
	      $ \alpha = |z| > 1$ and $\alpha_n = |z|^n.$
	      Looking at the \textit{limit of the sequence} we see it increasing arbitrarily, and so each
	      term is larger than the previous which means we can't ever satisfy our definition of a limit
	      since we \textit{can} choose some $m, n$ that would have some arbitrarily large difference between
	      them $> \epsilon.$ \qed

	\item Show that for any complex number $z \neq 1,$ we have
	      \[ 1 + z + \cdots + z^n = \frac{z^{n + 1} - 1}{z - 1}\]
	      If $|z| < 1,$ show that
	      \[ \lim_{n \to \infty} (1 + z + \cdots + z^n ) = \frac{1}{1 - z} \] \\

	      \textbf{Proof:}
	      Now we need to look at the \textit{limit of a sequence of partial sums} and test convergence.

	      \begin{align*}
		      S_1    & = 1                          \\
		      S_2    & = 1 + z                      \\
		      S_3    & = 1 + z + z^2                \\
		      \vdots & = \vdots                     \\
		      S_n    & = 1 + z + z^2 + \cdots + z^n
	      \end{align*}
	      Now play with the algebra and multiply both sides by $z$ for shits and giggles
	      \begin{align*}
		      zS_n       & = z + z^2 + \cdots + z^{n +1}       \\
		      zS_n - S_n & = z + z^2 + \cdots + z^{n +1} - S_n \\
		      S_n(z - 1) & = z + z^2 + \cdots + z^{n +1} - S_n \\
	      \end{align*}
	      The telescoping series starts to look more obvious
	      \begin{align*}
		      S_n(z - 1) & = z + z^2 + \cdots + z^{n +1} - (1 + z + z^2 + \cdots + z^n) \\
	      \end{align*}
	      Matching up the pairs we see the result we need
	      \begin{align*}
		      S_n(z - 1)     & = z + z^2 + \cdots + z^{n +1} - 1 - z - z^2 - \cdots - z^n)          \\
		      S_n(z - 1)     & = z^{n +1} - 1 + z - z + z^2 - z^2 + z^3 - z^3 + \cdots + z^n - z^n) \\
		      \therefore S_n & = \frac{z^{n +1} - 1}{z - 1}                                         \\
	      \end{align*}
	      And now we do as we said, \textit{take the limit of a sequence of partial sums}
	      \begin{align*}
		      \lim_{n \to \infty} S_n & = \lim_{n \to \infty} \frac{z^{n +1} - 1}{z - 1} \\
	      \end{align*}

	      And clearly this sequence diverges if $|z| > 1$ since the numerator grows unbounded over a fixed denominator.

	      Now consider if $|z| < 1$ and
	      \begin{align*}
		      \lim_{n \to \infty} S_n                                & = \lim_{n \to \infty} \frac{z^{n +1} - 1}{z - 1} \\
		                                                             & =  \frac{- 1}{z - 1}                             \\
		                                                             & =  \frac{1}{1 - z}                               \\
		      \therefore \lim_{n \to \infty} (1 + z + \cdots + z^n ) & = \frac{1}{1 - z}
	      \end{align*}
	      \qed

	\item Let $f$ be the function defined by
	      \[ f(z) = \lim_{n \to \infty} \frac{1}{1 + n^2 z} \]

	      Show that $f$ is the characteristic function of the set $\{0\},$ that is, $f(0) = 1,$ and $f(z) = 0$ if $z \neq 0.$

	      \textbf{Proof:}
	      Start by plugging in:
	      \begin{align*}
		      f(z) & = \lim_{n \to \infty} \frac{1}{1 + n^2 z} \\
		      f(0) & = \lim_{n \to \infty} \frac{1}{1 + n^2 0} \\
		      f(0) & = 1
	      \end{align*}

	      Now consider all other cases:

	      \begin{align*}
		      f(+\infty) & = \lim_{n \to \infty} \frac{1}{1 + n^2 \infty}     \\
		      f(+\infty) & = 0                                                \\
		      f(-\infty) & = \lim_{n \to -\infty} \frac{1}{1 + n^2 (-\infty)} \\
		      f(-\infty) & = 0                                                \\
	      \end{align*}

	      Then consider $b < 1 < c$:

	      \begin{align*}
		      f(c) & = \lim_{n \to \infty} \frac{1}{1 + n^2 c} \\
		      f(c) & = 0                                       \\
		      f(b) & = \lim_{n \to \infty} \frac{1}{1 + n^2 b} \\
		      f(b) & = 0                                       \\
	      \end{align*}

	      And so the function $f$ satisfies the definition and so can be called the characteristic function of the set $\{0\}$.
	      \qed

	\item For $|z| \neq 1$ show that the following limit exists:
	      \[ f(z) = \lim_{n \to \infty} \left( \frac{z^n - 1}{z^n + 1} \right) \]

	      Is it possible to define $f(z)$ when $|z| = 1$ in such a way to make $f$ continuous?

	      \textbf{Proof:}
	      Take the limits for the cases of $z$.

	      First for $z > 1$:
	      \begin{align*}
		      f(z) & = \lim_{n \to \infty} \left( \frac{z^n - 1}{z^n + 1} \right)                                     \\
		           & = \lim_{n \to \infty} \frac{z^n}{z^n} \left( \frac{1 - \frac{1}{z^n}}{1 + \frac{1}{z^n}} \right) \\
		           & = \lim_{n \to \infty} \left( \frac{1 - \frac{1}{z^n}}{1 + \frac{1}{z^n}} \right)                 \\
		           & = \left( \frac{1 - 0}{1 + 0} \right)                                                             \\
		           & = 1
	      \end{align*}

	      For $z < 1$:
	      \begin{align*}
		      f(z) & = \lim_{n \to \infty} \left( \frac{z^n - 1}{z^n + 1} \right)                                     \\
		           & = \lim_{n \to \infty} \frac{z^n}{z^n} \left( \frac{1 - \frac{1}{z^n}}{1 + \frac{1}{z^n}} \right) \\
		           & = \lim_{n \to \infty} \left( \frac{1 - \frac{1}{z^n}}{1 + \frac{1}{z^n}} \right)                 \\
		           & = \lim_{n \to \infty} \left( \frac{- \frac{1}{z^n}}{\frac{1}{z^n}} \right)                       \\
		           & = \lim_{n \to \infty} \left( \frac{- z^n}{{z^n}} \right)                                         \\
		           & = -1
	      \end{align*}

	      And so the limit exists for $|z| \not = 1.$

	      \emph{Can we make the function continuous?} No, because we cannot define a value there to satisfy the definition of continuity. The limit at
	      $|z| = 1$ oscillates between $-1$ and $1$, so the limit does not exist at the point because if it did, the limit at that point would be unique.

	      \qed

	\item Let \[ f(z) = \lim_{n \to \infty} \frac{z^n}{1 + z^n} \]
	      \begin{enumerate}
		      \item What is the domain of definition of $f,$ that is, for which complex numbers $z$ does the limit exist?

		            \textbf{Proof:}
		            The domain clearly exists for the cases of $|z| < 1$:
		            \begin{align*}
			            f(z) & = \lim_{n \to \infty} \frac{z^n}{1 + z^n} \\
			                 & = \frac{0}{1 + 0}                         \\
			                 & = 0
		            \end{align*}
		            And then for $z > -1$:
		            \begin{align*}
			            f(z) & = \lim_{n \to \infty} \frac{z^n}{1 + z^n} \\
			                 & = \frac{-\infty}{1 + -\infty}             \\
			                 & = 1
		            \end{align*}

		            Now to consider $z = -1$. Here we have an issue of never settling on a sign which is a divergent sequence.
		            That is enough to show that a singularity exists at this point,
		            which is to say the limit does not exist at $z = -1$.

		            $\therefore$ The domain of definition is $(-\infty, -1) \cup (-1, \infty)$.
		            \qed

		      \item Give explicitly the values of $f(z)$ for the various $z$ in the domain of $f.$

		            \textbf{Proof:}
		            See above problem. \qed

	      \end{enumerate}

	\item Show that the series
	      \[ \sum_{n = 1}^{\infty} \frac{z^{n - 1}}{(1 - z^n )(1 - z^{n + 1})} \]
	      Converges to
	      \[ \frac{1}{(1 - z)^2 } \;\;\; |z| < 1 \]
	      and
	      \[\frac{1}{z(1 - z)^2 } \;\;\; |z| > 1\]

	      Prove that the convergence is uniform for $|z| \leq c < 1$ in the first case, and $|z| \geq b > 1$ in the second case.

		      [\textit{Hint:} multiply and divide each term by $1 - z,$ and do a partial fraction decomposition, getting a telescoping effect. ]

	      \textbf{Proof:}

	      For this problem, just follow along with Lang's suggestion and watch the telescope emerge. Let's look at the $n$th term for $z$ and see
	      if there's something there.

	      \begin{align*}
		      z_n                               & = \frac{z^{n - 1}}{(1 - z^{n})(1 - z^{n + 1})}               \\
		      z_n \cdot \frac{(1 - z)}{(1 - z)} & = \frac{z^{n - 1}}{(1 - z^{n})(1 - z^{n + 1})}               \\
		      z_n (1 - z)                       & = (1 - z) \cdot \frac{z^{n - 1}}{(1 - z^{n})(1 - z^{n + 1})} \\
		      z_n (1 - z)                       & =  \frac{(1 - z)z^{n - 1}}{(1 - z^{n})(1 - z^{n + 1})}       \\
		      z_n (1 - z)                       & =  \frac{z^{n - 1} - z^n}{(1 - z^{n})(1 - z^{n + 1})}        \\
	      \end{align*}

	      Now use \textbf{partial fraction decomposition}

	      \begin{align*}
		      \frac{z^{n - 1} - z^n}{(1 - z^{n})(1 - z^{n + 1})} & =  \frac{z^{n - 1}}{(1 - z^{n})} - \frac{z^n}{(1 - z^{n + 1})} \\
	      \end{align*}

	      Then

	      \begin{align*}
		      z_n (1 - z)           & =  \frac{z^{n - 1}}{(1 - z^{n})} - \frac{z^n}{(1 - z^{n + 1})}                              \\
		      \therefore \;\;\; z_n & =  \frac{1}{(1 - z)} \left( \frac{z^{n - 1}}{1 - z^{n}} - \frac{z^n}{1 - z^{n + 1}} \right) \\
	      \end{align*}

	      Now compute $z_1$ and $z_2$ with this new expression for $z_n$

	      \begin{align*}
		      z_1 & = \frac{1}{(1 - z)} \left( \frac{z^{1 - 1}}{1 - z^{1}} - \frac{z^1}{1 - z^{1 + 1}} \right) \\
		          & = \frac{1}{(1 - z)} \left( \frac{1}{1 - z} - \frac{z}{1 - z^{2}} \right)                   \\
		          & = \frac{1}{(1 - z)^2} - \frac{z}{(1 - z)(1 - z^{2})}                                       \\
		      \\
		      z_2 & = \frac{1}{(1 - z)} \left( \frac{z^{2 - 1}}{1 - z^{2}} - \frac{z^2}{1 - z^{2 + 1}} \right) \\
		          & = \frac{1}{(1 - z)} \left( \frac{z}{1 - z^2} - \frac{z^2}{1 - z^{3}} \right)               \\
		          & = \frac{z}{(1 - z)(1 - z)^2} - \frac{z}{(1 - z)(1 - z^{2})}                                \\
	      \end{align*}

	      Now sum these and notice the telescoping pattern emerging between them

	      \begin{align*}
		      z_1 + z_2 & = \frac{1}{(1 - z)^2} - \frac{z}{(1 - z)(1 - z^{2})} + \frac{z}{(1 - z)(1 - z)^2} - \frac{z}{(1 - z)(1 - z^{2})} \\
		      z_1 + z_2 & = \frac{1}{(1 - z)^2} - \frac{z}{(1 - z)(1 - z^{2})}                                                             \\
	      \end{align*}

	      The terms will match up and we are left with

	      \begin{align*}
		      z_1 + z_2 + \cdots + z_n & = \frac{1}{(1 - z)^2}                                                                                     \\
		                               & - \frac{z}{(1 - z)(1 - z^{2})} + \frac{z}{(1 - z)(1 - z)^2}                                               \\
		                               & - \frac{z^2}{(1 - z)(1 - z^{2})} + \frac{z^2}{(1 - z)(1 - z^{2})} + \cdots                                \\
		                               & - \frac{z^n-1}{(1 - z)(1 - z^{n})} +\frac{z^n-1}{(1 - z)(1 - z^{n})} - \frac{z^n}{(1 - z)(1 - z^{n + 1})} \\
		                               & = \frac{1}{(1 - z)^2} - \frac{z^n}{(1 - z)(1 - z^{n + 1})}                                                \\
	      \end{align*}

	      Clearly for $|z| < 1$ and taking the $\lim n \to \infty$ the last term goes to $0$ and we are left
	      with the desired result. \textit{This concludes the case for $|z| < 1$.}

	      Now consider the case for $|z| > 1$

	      \begin{align*}
		      z_1 + z_2 + \cdots + z_n & = \frac{1}{(1 - z)^2} - \frac{z^n}{(1 - z)(1 - z^{n + 1})}                 \\
		                               & = \frac{1}{(1 - z)^2} - \frac{z^n}{(1 - z)(1 - z)(z^n)}                    \\
		                               & = \frac{1}{(1 - z)^2} - \frac{1}{(1 - z)(1 - z)}                           \\
		                               & = \frac{1}{(1 - z)^2} - \frac{1}{(z^2 - z)}                                \\
		                               & = \frac{1}{(1 - z)^2} - \frac{1}{z(z - 1)}                                 \\
		                               & = \frac{z(z - 1)}{z(z - 1)(1 - z)^2} - \frac{(1 - z)^2}{z(z - 1)(1 - z)^2} \\
		                               & = \frac{z(z - 1) - (1 - z)^2}{z(z - 1)(1 - z)^2}                           \\
		                               & = \frac{z^2 - z  - [(1 - z)(1 - 2)]}{z(z - 1)(1 - z)^2}                    \\
		                               & = \frac{z^2 - z  - (1 - 2z + z^2)}{z(z - 1)(1 - z)^2}                      \\
		                               & = \frac{z^2 - z  - 1 + 2z - z^2}{z(z - 1)(1 - z)^2}                        \\
		                               & = \frac{z - 1}{z(z - 1)(1 - z)^2}                                          \\
		                               & = \frac{1}{z(1 - z)^2}                                                     \\
	      \end{align*}
	      \qed

\end{enumerate}
\newpage