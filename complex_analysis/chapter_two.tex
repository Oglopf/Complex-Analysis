\chapter{Power Series}
We've already been dancing around with these in some previous proofs, now let's really dig in. 
\section{Formal Power Series}
\begin{defn}[\textbf{Formal Power Series}]
  Using a neutral letter $T$:

  \[\sum_{n = 0}^{\infty}a_n T^n = a_0 + a_1 T + a_2 T^2 + \cdots \]

\end{defn}
The important part of this definition are the \textit{coefficients} $a_0, a_1, a_2, ...$ which we 
take as complex numbers.
\begin{itemize}
  \item You could think of this as \textit{a map from the integers} $ \geq 0 $ \textit{to the complex numbers.}
  \[n \mapsto a_n \]
\end{itemize}

Main points: 
  \begin{itemize}
    \item if you wanna compose functions and maps, do it term by term with their series expansions.
    \item Often when doing all this you can even get a telescoping series and then arrive at a closed form of the expression. 
  \end{itemize}

For the below definitions refer to the \textbf{formal expression of a power series:}

\begin{align*}
  f(x) &= \sum_{n = 0}^{\infty} a_n T^n \\
  &= a_0 + a_1 T + a_2 T^2 + \cdots
\end{align*}

\begin{defn}[\textbf{Constant Term}]
  The leading term of $f$ denoted $a_0$.
\end{defn}

\begin{defn}[\textbf{Order} $r$ \textbf{of} $f$]
  $r$ is the smallest integer $n$ such that $a_n \not = 0$
  \[r = \text{ord}\; f\]
\end{defn}

\begin{thm}
  Any power series has order $0$ if and only if it starts with \textit{constant term} $\not = 0.$
\end{thm}

\begin{thm}
  \text{ord}\; fg = \text{ord}\; f + \text{ord}\; g
\end{thm}

\begin{defn}[\textbf{Inverse}]
  Let $g = \sum b_n T^n$
  \[gf = 1\]
\end{defn}
This leads to the following Theorem:
\begin{thm}
  If f has a non-zero constant term, then f has an inverse as a power series.
\end{thm}

\subsection{Proofs}

\begin{enumerate}

  \item Give the terms of order $\leq 3$ in the power series:
  The trick for all of these is to just use their power series expansions or use ones we've found and stuff more $f$s into them 
  to get out more that we need.
  \begin{enumerate}
    \item $e^z \sin(z)$
    \item $\sin z \cos z$
    \item $\frac{e^z - 1}{z}$
    \item $\frac{e^z - \cos z}{z}$
    \item $\frac{1}{\cos z}$
    \item $\frac{\cos z}{\sin z}$
    \item $\frac{\sin z}{\cos z}$
    \item $\frac{e^z}{\sin z}$
  \end{enumerate}
  
  \item Let $f(z) = \sum a_n z^n$.

  Define $f(-z) = \sum a_n (-z)^n = \sum a_n(-1)^n z^n$.

  Define $f$ as \textbf{even} if $a_n = 0$ for $n$ odd.

  Define $f$ as \textbf{odd} if $a_n = 0$ for $n$ even.

  Verify that $f$ is even if and only if $f(-z) = f(z)$ and $f$ is odd if and only if $f(-z) = -f(z).$

  \textbf{Proof}:
  Use power series expansions. Proving the \textbf{even} case first, but the same type of argument will work for the odd.

  If $f$ is \textbf{even} then $n$ is odd, so:
  \begin{align*}
    f(-z) &= \sum a_n z^n (-1)^n \\
    &= a_1 z (-1) + a_2 z^2 (-1)^2 + a_3 z^3 (-1)^3 \\
    &+ a_4 z^4 (-1)^4 + a_5 z^5 (-1)^5 + a_6 z^6 (-1)^6 + \cdots + a_n z^n (-1)^n \\
  \end{align*}

  Recall that since this is \textbf{even} then all the $a_n = 0$ for odd $n$ which leaves us with (note $k \in \mathbb{N}$ here):
  \begin{align*}
    f(-z) &= a_2 z^2 (-1)^2 + a_4 z^4 (-1)^4 + a_6 z^6 (-1)^6 + \cdots + a_n z^{2k} (-1)^{2k} \\
  \end{align*} 
  Noticing that all of the factors of $(-1)$ have the form $(-1)^{2k}$ we see:
  \begin{align*}
    f(-z) = f(z)
  \end{align*}
  Conversely, if we assume $f(-z) = f(z)$ then $f$ is \textbf{even} simply by definition, which completes the proof 
  for the \textbf{even} case. 
  
  A similar argument can be applied for the \textbf{odd} case.
  \qed

  \item Define the \textbf{Bernoulli numbers} $B_n$ by the power series:

  \begin{align*}
    \frac{z}{e^z - 1} = \sum_{n = 0}^{\infty}\frac{B_n}{n\,!}z^n \\
  \end{align*}

  Prove the recursion formula:

  \begin{align*}
    \frac{B_0}{n \, ! \; 0!} + \frac{B_1}{(n - 1)! \; 1!} + \cdots + \frac{B_{n - 1}}{1! \; (n - 1)\,!} + = 
    \begin{cases}
      1 \;\;\; \text{   if } n = 1. \\
      0 \;\;\; \text{   if } n > 1. \\
    \end{cases}
  \end{align*}
  Then $B_0 = 1.$
  
  Compute $B_1, B_2, B_3, B_4.$
  
  Show that $B_n = 0$ if $n$ is odd and $n \not = 1.$
  
  \item Show that:

  \begin{align*}
    \frac{z}{2} \frac{e^{z/2} + e^{-z/2}}{e^{z/2} - e^{-z/2}} = \sum_{n = 0}^\infty \frac{B_n}{(2n)!}z^{2n} \;. \\
  \end{align*}

  Replace $z$ by $2\pi i z$ to show that:

  \begin{align*}
    \pi z \cot(\pi z) = \sum_{n = 0}^\infty (-1)^n \frac{(2\pi)^{2n}}{(2n)!} z^{2n} B_{2n} \;.
  \end{align*}

  \item Express the power series for $\tan z , \frac{z}{\sin z}, z \cot z ,$ in terms of \textbf{Bernoulli numbers}.

  \item (\textbf{Difference Equation}) given complex numbers $a_0, a_1, u_1, u_2 \;$ define $a_n$ for $2 \leq n$ by:

  \begin{align*}
    a_n = u_1 a_{n - 1} +u_2 a_{n - 2} \;.
  \end{align*}

  If we have a factorization: \textbf{STOPPED}
\end{enumerate}

\section{Convergent Power Series}
Let ${ \{ z_n \} }$ be a \emph{sequence of complex numbers}.

Consider the series:
\begin{align*}
  \sum_{n = 1}^{\infty} z_n
\end{align*}

\begin{defn}[\textbf{Partial Sum of a Sequence of Complex Numbers}]
  \begin{align*}
    s_n = \sum_{k = 1}^{n} z_k = z_1 + z_2 + \cdots z_n
  \end{align*}
\end{defn}

\begin{defn}[\textbf{Convergent Series}]
  The series $s_n$ converges if there is some $w \in \mathbb{C}$ such that the following limit exists:
  \begin{align*}
    \lim_{n \to \infty}s_n = w
  \end{align*}
\end{defn}

In which case we say that $w$ is equal to the \emph{sum of the series.}

Formally:

\begin{defn}[\textbf{Sum of the Series}]
  \begin{align*}
    w = \sum_{n = 1}^{\infty} z_n
  \end{align*}
\end{defn}

Now consider series of functions and deal with questions of \emph{uniformity.}

Suppose that $S$ is a set and $f$ a bounded function on $S$.

\begin{defn}[\textbf{Sup norm}]
  Let sup be the least upper bound, then:
  \begin{align*}
    \| f \|_S = \| f \| = \sup_{z \in S} | f(z) | \\
  \end{align*}
\end{defn}

Now, let $\{ f_n \}$ with $n \in \mathbb{N}$ be a \emph{sequence of functions} on $S$.

\begin{defn}[\textbf{Uniform Convergence}]
  The sequence $ \{ f_n \}$ converges uniformly on $S$ if there exists a 
  function $f$ on $S$ satisfying the following properties.

  Given $\epsilon$, there exists $N$ such that if $n \geq N$ then:

  \begin{align*}
    \| f_n - f \| < \epsilon \\
  \end{align*}
\end{defn}

\begin{defn}[\textbf{Cauchy Sequence}]
  We call $\{f_n\}$ Cauchy if given $\epsilon$ there exists $N$ such that if $m, n \geq N$ then:
  \begin{align*}
    \| f_n - f_m \| < \epsilon
  \end{align*}
\end{defn}

It is important to note that if a sequence $\{f_n\}$ of functions is Cauchy, then it converges 
uniformly. You should be able to prove this to yourself (think about what you get if you're cauchy, hint: it 
is small!)

Now, keeping in mind we are working wit sequences of functions at this point, we define:

\begin{defn}[\textbf{Partial Sum of a Sequence of Complex Functions}]
  \begin{align*}
    s_n = \sum_{k = 1}^{n} f_k = f_1 + f_2 + \cdots f_n
  \end{align*}
\end{defn}

This above series converges \emph{uniformly} if the sequence of partial sums $\{s_n\}$ converges uniformly.

\section{Relations Between Formal and Convergent Series}
\section{Analytic Functions}
\section{Differentiation of Power Series}
\section{The Inverse and Open Mapping Theorems}
\section{The Local Maximum Modulus Principle}