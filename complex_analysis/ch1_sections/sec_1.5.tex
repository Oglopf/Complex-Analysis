\section{Complex Differentiability}
\underline{{\textit{\textbf{There are no exercises in this section.}}}}

\begin{itemize}
	\item Let $U$ be an open set.
	\item Let $f$ be a function on $U.$
\end{itemize}

\begin{defn}[$f$ \textbf{Complex Differentiable at} $z$]
	If the limit exists:
	\begin{align*}
		\lim_{h \to 0} \frac{f(z + h) - f(z)}{h} \\
	\end{align*}


	denoted by $f'(z)$ or $df/dz.$
\end{defn}
\begin{itemize}
	\item \textit{Note:} If $f$ is differentiablre at $z$ then $f$ is continuous at $z.$
\end{itemize}

All that really matters in this section
is that all the usual rules for sums, products, quotients, and functions of functions are the same
wrt complex Differentiability as they were with real Differentiability.

\subsection{Holomorphic Function}
A function $f$ defined on an open set $U$ is said to be \textbf{differentiable} if it is differentiable at every point.
\begin{itemize}
	\item Also say that $f$ is \textbf{Holomorphic} on $U.$
	\item Holomorphic is usually used to specify \textit{complex} differentiability as distinguised from \textit{real} differentiability.
\end{itemize}


\begin{defn}[\textbf{Holomorphic Isomorphism}]
	A holomorphic function
	\begin{align*}
		f: U \to V
	\end{align*}
	From an open set into another open set is a \textbf{holomorphic isomorphism} if there exists a holomorphic function
	\begin{align*}
		g: V \to U
	\end{align*}
	such that $g$ is the inverse of $f.$ That is;
	\begin{align*}
		g \circ f & = id_u \,\,\,\,
		and \,\,\,\,
		f \circ g = id_v
	\end{align*}
\end{defn}

\begin{defn}[\textbf{Holomorphic Automorphism}]
	A holomorphic isomorphism of an open set $U$ with itself.
\end{defn}
\newpage