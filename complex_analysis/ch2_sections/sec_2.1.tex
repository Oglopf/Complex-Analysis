We've already been dancing around with these in some previous proofs, now let's really dig in. 
\section{Formal Power Series}
\begin{defn}[\textbf{Formal Power Series}]
  Using a neutral letter $T$:

  \[\sum_{n = 0}^{\infty}a_n T^n = a_0 + a_1 T + a_2 T^2 + \cdots \]

\end{defn}
The important part of this definition are the \textit{coefficients} $a_0, a_1, a_2, ...$ which we 
take as complex numbers.
\begin{itemize}
  \item You could think of this as \textit{a map from the integers} $ \geq 0 $ \textit{to the complex numbers.}
  \[n \mapsto a_n \]
\end{itemize}

Main points: 
  \begin{itemize}
    \item if you wanna compose functions and maps, do it term by term with their series expansions.
    \item Often when doing all this you can even get a telescoping series and then arrive at a closed form of the expression. 
  \end{itemize}

For the below definitions refer to the \textbf{formal expression of a power series:}

\begin{align*}
  f(x) &= \sum_{n = 0}^{\infty} a_n T^n \\
  &= a_0 + a_1 T + a_2 T^2 + \cdots
\end{align*}

\begin{defn}[\textbf{Constant Term}]
  The leading term of $f$ denoted $a_0$.
\end{defn}

\begin{defn}[\textbf{Order} $r$ \textbf{of} $f$]
  $r$ is the smallest integer $n$ such that $a_n \not = 0$
  \[r = \text{ord}\; f\]
\end{defn}

\begin{thm}
  Any power series has order $0$ if and only if it starts with \textit{constant term} $\not = 0.$
\end{thm}

\begin{thm}
  \text{ord}\; fg = \text{ord}\; f + \text{ord}\; g
\end{thm}

\begin{defn}[\textbf{Inverse}]
  Let $g = \sum b_n T^n$
  \[gf = 1\]
\end{defn}
This leads to the following Theorem:
\begin{thm}
  If f has a non-zero constant term, then f has an inverse as a power series.
\end{thm}
\newpage