\chapter{Power Series}
We've already been dancing around with these in some previous proofs, now let's really dig in. 
\section{Formal Power Series}
\begin{defn}[\textbf{Formal Power Series}]
  Using a neutral letter $T$
  \[\sum_{n = 0}^{\infty}a_n T^n = a_0 + a_1 T + a_2 T^2 + \cdots \]
\end{defn}
The important part of this definition are the \textit{coefficients} $a_0, a_1, a_2, ...$ which we 
take as complex numbers.
\begin{itemize}
  \item You could think of this as \textit{a map from the integers} $ \geq 0 $ \textit{to the complex numbers.}
  \[n \mapsto a_n \]
\end{itemize}

The whole big point is this: 
  \begin{itemize}
    \item if you wanna compose functions and maps, do it term by term with their series.
  \end{itemize}
Often when doing all this you can even get a telescoping series and then arrive at a closed form of the expression.

For the below definitions refer to the formal expression:
\begin{align*}
  f(x) &= \sum_{n = 0}^{\infty} a_n T^n \\
  &= a_0 + a_1 T + a_2 T^2 + \cdots
\end{align*}
\begin{defn}[\textbf{Constant Term}]
  The leading term of $f$ denoted $a_0$.
\end{defn}

\begin{defn}[\textbf{Order} $r$ \textbf{of} $f$]
  $r$ is the smallest integer $n$ such that $a_n \not = 0$
  \[r = \text{ord}\; f\]
\end{defn}

\begin{thm}
  Any power series has order $0$ if and only if it starts with \textit{constant term} $\not = 0.$
\end{thm}

\begin{thm}
  \text{ord}\; fg = \text{ord}\; f + \text{ord}\; g
\end{thm}

\begin{defn}[\textbf{Inverse}]
  Let $g = \sum b_n T^n$
  \[gf = 1\]
\end{defn}
This leads to the following Theorem:
\begin{thm}
  If f has a non-zero constant term, then f has an inverse as a power series.
\end{thm}

\subsection{Proofs}
\begin{enumerate}
  \item Give the terms of order $\leq 3$ in the power series:
  The trick for all of these is to just use their power series expansions or use ones we've found and stuff more $f$s into them 
  to get out more that we need.
  \begin{enumerate}
    \item $e^z \sin(z)$
    \item $\sin z \cos z$
    \item $\frac{e^z - 1}{z}$
    \item $\frac{e^z - \cos z}{z}$
    \item $\frac{1}{\cos z}$
    \item $\frac{\cos z}{\sin z}$
    \item $\frac{\sin z}{\cos z}$
    \item $\frac{e^z}{\sin z}$
  \end{enumerate}
  
  \item Let $f(z) = \sum a_n z^n$. \\
  Define $f(-z) = \sum a_n (-z)^n = \sum a_n(-1)^n z^n$. \\
  Define $f$ as \textbf{even} if $a_n = 0$ for $n$ odd. \\
  Define $f$ as \textbf{odd} if $a_n = 0$ for $n$ even. \\
  Verify that $f$ is even if and only if $f(-z) = f(z)$ and $f$ is odd if and only if $f(-z) = -f(z).$ \\

  \textbf{Proof}: \\
  Use power series expansions. Proving the \textbf{even} case first, but the same type of argument will work for the odd.

  If $f$ is \textbf{even} then $n$ is odd, so:
  \begin{align*}
    f(-z) &= \sum a_n z^n (-1)^n \\
    &= a_1 z (-1) + a_2 z^2 (-1)^2 + a_3 z^3 (-1)^3 \\ 
    &+ a_4 z^4 (-1)^4 + a_5 z^5 (-1)^5 + a_6 z^6 (-1)^6 + \cdots + a_n z^n (-1)^n \\
  \end{align*}
  Recall that since this is \textbf{even} then all the $a_n = 0$ for odd $n$ which leaves us with (note $k \in \mathbb{N}$ here):
  \begin{align*}
    f(-z) &= a_2 z^2 (-1)^2 + a_4 z^4 (-1)^4 + a_6 z^6 (-1)^6 + \cdots + a_n z^{2k} (-1)^{2k} \\
  \end{align*} 
  Noticing that all of the factors of $(-1)$ have the form $(-1)^{2k}$ we see:
  \begin{align*}
    f(-z) = f(z)
  \end{align*}
  Conversely, if we assume $f(-z) = f(z)$ then $f$ is \textbf{even} simply by definition. A similar argument can be applied for the \textbf{odd} case.
  \qed
\end{enumerate}

\section{Convergent Power Series}
\section{Relations Between Formal and Convergent Series}
\section{Analytic Functions}
\section{Differentiation of Power Series}
\section{The Inverse and Open Mapping Theorems}
\section{The Local Maximum Modulus Principle}