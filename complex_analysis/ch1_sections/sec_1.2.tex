\section{Polar Form}

Let $z = x +iy.$

\begin{defn}[\textbf{Polar Coordinates}]
	An ordered pair $(r, \theta)$ with $r = radius$ and $\theta$ rotating from the $x$-axis such that:
	\begin{enumerate}
		\item $r \in \mathbb{R}$ and $r = |z| = \sqrt{x^2 + y^2}.$
		\item $\theta  \in [0, 2\pi].$
	\end{enumerate}
\end{defn}
\begin{defn}[\textbf{Polar Form}]
	\begin{align*}
		re^{i\theta} &= r\cos{\theta} + ir\sin{\theta} \\
		&= x + iy \\
		\therefore \;\; &re^{i\theta} \in \mathbb{C}
	\end{align*}
\end{defn}
Note that:
\[x = r\cos{\theta}, \;\;\; y = r\sin{\theta}\]
\[\theta = \cos^{-1}{ \left( \frac{x}{r} \right)  }, \;\;\;\; \theta = \sin^{-1} \left( { \frac{y}{r} } \right) \]
\subsection{Factors of Pi}
\begin{defn}[\textbf{Factors of Pi}]
	"If you don't know your factors of pi you don't know squat" - Big Rick Feynman

	\begin{center}
		$n \in \mathbb{N} \;\;\;\;\;\;\;\; z, w \in \mathbb{C}$
	\end{center}
	\begin{align*}
		e^{0} &= 1 + i0 & e^{0} &= (1, 0) \\
		e^{\sfrac{i\pi}{6}} &= \frac{\sqrt{3}}{2} + i\frac{1}{2} & e^{\sfrac{i\pi}{6}} &= \left( \frac{\sqrt{3}}{2}, \frac{1}{2} \right)  \\
		e^{\sfrac{i\pi}{4}} &= \frac{\sqrt{2}}{2} + i\frac{\sqrt{2}}{2} & e^{\sfrac{i\pi}{4}} &= \left( \frac{\sqrt{2}}{2}, \frac{\sqrt{2}}{2} \right) \\
		e^{\sfrac{i\pi}{3}} &= \frac{1}{2} + i\frac{\sqrt{3}}{2} & e^{\sfrac{i\pi}{3}} &= \left(\frac{1}{2}, \frac{\sqrt{3}}{2} \right) \\
		e^{\sfrac{i\pi}{2}} &= 0 +  i & e^{\sfrac{i\pi}{2}} &= (0, 1) \\
		e^{i\pi} &= -1 + i0 & e^{i\pi} &= (-1, 0) \\
		e^{2i\pi} &= 1 + i0 & e^{2i\pi} &= (1, 0) \\
		\\
		e^{2n\pi i} &= 1 &\text{If} \;\; e^z &= e^w \;\: \text{then} \;\; z = w + 2k\pi i \\
	\end{align*}
	Now take sums and multiples to build more factors from these.
\end{defn}
 
\begin{thm}
	Let $\theta, \varphi \in \mathbb{R}$ then:
	\[e^{i\theta + i\varphi} = e^{i\theta}e^{i\varphi}\]
\end{thm}
\begin{thm}
	Let $\alpha, \beta \in \mathbb{C}$ then:
	\[e^{\alpha + \beta} = e^{\alpha}e^{\beta}\]
\end{thm}
\begin{thm}[1.2 and 1.3 together]
	Let $z_1 = r_1 e^{i\theta}$ and $z_2 = r_2 e^{i\varphi}$ then:

	\[z_1 \cdot z_2 = r_1 r_2 e^{i(\theta + \varphi)} \]

	i.e. multiply the absolute values and add the angles.
\end{thm}
\newpage